\chapter{確率分布}

\begin{itemize}
    \item 観測値$\bm{x}_2, ... \bm{x}_N$が与えられた時の確率変数$p(\bm{x})$をモデル化する問題を密度推定という.
    \item 観測されたデータ集合を生成しうる確率分布は無限に考えられるため不良設定.
\end{itemize}

\section{二値変数}

\subsection{ベルヌーイ分布}
確率変数$x$が$x \{0, 1\}$の場合を考えてみる. コイン投げとか.$x=1$となる確率を$\mu$とすると

$$p(x=1|\mu) = \mu$$
$$p(x=0|\mu) = 1 - \mu$$

となる. したがって確率分布は

$$p(x|\mu) = \mu^x(1-\mu)^{1-x}$$

となる. これをベルヌーイ分布という.
$x=1$の時は$\mu^x$が1になる. 逆に$x=0$の時は$(1-\mu)^{1-x}$が1になる.

教科書の式(2.3)と式(2.4)で平均と分散も確認.

次に尤度関数を考えたい. そこでxの観測値のデータ集合$D = \{x_1, ..., x_n\}$があるとする.

すると, 尤度関数は同時確率で表されるため

$$p(D|\mu) = \prod_{n=1}^{N}p(x_n|\mu) = \prod_{n=1}^{N}\mu^{x_n}(1-\mu)^{1-x_n}$$

となる.

対数尤度関数は

$$\ln p(D|x) = \sum_{n=1}^{N} \{\ln x_n\mu + (1-x_n)\ln (1-\mu) \}$$

となる.

この時, $\sum_n x_n$は十分統計量となっている.

十分統計量とは, ある確率分布に対して, その確率分布がもつパラメータ$\theta$がわからなくても, 「その他の値」から関数分布を推定できるような「その他の値」のこと.

例えば, ベルヌーイ分布はパラメータ$\mu$がわからなくとも$\sum_{n} x_n$がわかれば確率分布の形が推定できる.
このサイトがわかりやすい. (https://abicky.net/2014/03/03/202054/)

実際に教科書の式(2.7)からわかるように, 最尤推定の結果$\mu_{ML}$は$\sum_{n=1}^{N} x_n$を使って表される.

\subsection{二項分布}

ベルヌーイ分布は一回試行の確率分布だったため, N回試行に拡張した確率分布である二項分布を見てみる.

$N$回試行して$m$回表が出たとするとその確率は$\mu^{m}(1-\mu)^{N-m}$と表すことができる. 

N回試行の中でm回表が出るパターンは$\frac{N!}{(N-m)!m!}$となる.

よって, N回試行してm回表が出る確率は, ある1パターンにおけるm回表が出る確率に全パターン数をかけて

$$p(m|N, \mu) = \frac{N!}{(N-m)!m!}\mu^{m}(1-\mu)^{N-m}$$

と表すことができる.

平均と分散は教科書を確認.

\subsection{ベータ分布}

ベルヌーイ分布や二項分布は最尤推定を使うと表になる割合が出てくることがわかった.
これは過学習しやすい. 3回中3回が表のコイン投げのデータを使って最尤推定すると$\mu_{ML}=1$となった. (教科書確認)

そこでベイズ的考えを導入する. そのためには事前分布を導入しなければならない. ここで都合のいい事前分布について考えてみる. 

事後分布は尤度と事前分布の積に比例するのであった.

例えば尤度関数が$\mu^x(1-\mu)^{1-x}$のような形になっているなら事前分布を$\mu$もしくは$(1-\mu)$のべき乗になるように選べば, 事後分布は$\mu^{?}(1-\mu)^{?}$といった形になるはず.

これは共益性と呼ばれ色々と都合がいいっぽい. どう都合がいいのかはまだ書かれていない.

ベルヌーイ分布と二項分布に対して共益性のあるベータ分布を導入する.

$$Beta(\mu|a, b) = \frac{\Gamma (a+b)}{\Gamma (a) \Gamma (b)}\mu^{a-1}(1-\mu)^{b-1} $$

係数は正規化されることを保証している.

平均と分散は教科書で確認.

パラメータ$a$, $b$は$\mu$は決めるパラメータなのでハイパラメータと呼ばれる.

二項分布の$\mu$の事後分布は二項分布の尤度関数とベータ分布に比例するので

$$ p(\mu | m, N, a, b) \propto \mu^m(1-\mu)^{N-m} \times \frac{\Gamma (a+b)}{\Gamma (a) \Gamma (b)}\mu^{a-1}(1-\mu)^{b-1} $$

$\mu$に関係のない$\frac{\Gamma (a+b)}{\Gamma (a) \Gamma (b)}$を除外すると

\begin{eqnarray*}
    p(\mu | m, N, a, b) &\propto& \mu^m(1-\mu)^{N-m} \times \mu^{a-1}(1-\mu)^{b-1} \\
                        &\propto& \mu^{m+a-1}(1-\mu)^{N-m+b-1}
\end{eqnarray*}

となる. これもベータ分布とみなせるため, 正規化定数を合わせて

$$ p(\mu | m, l, a, b) = \frac{\Gamma (m+a+l+b)}{\Gamma (m+a) \Gamma (l+b)} \mu^{m+a-1}(1-\mu)^{N-m+b-1}$$

となる.

事前分布から事後分布を求めるには$a$の値を$m$だけ, $b$の値は$l$だけずらせば良いことがわかる.
よって$a$を$m$, $b$を$l$に置き換えたら再び事前分布のように振る舞えるため, 逐次学習を行うことができる.

教科書の図2.2をみるとサンプル数(a+b)が多ければ多いほど分散が小さくなる(確信度が高くなる).
この性質はベイズ推定で平均的に成り立つがもちろん平均的であって全てに対して成り立つわけではない.

\section{多値変数}

二値の場合を見てきたので, 今度はもっと拡張して多値の場合を見たい.
ここで相互に排他的なK個の値を取りうる離散値の確率分布を考えたい.

観測値の表し方はよくOneHotベクトルが用いられる.
要素を$k$と表すとすると$\{0, 1, ..., 5\}$の値を取りうる中で3が観測された場合は$x_3=1$となり, OneHotベクトルは$\bm{x} = \{ 0, 0, 0, 1, 0, 0 \}$と表される.

$x_k = 1$になる確率をパラメータ$\mu_k$で表すと

$$ p(\bm{x}|\bm{\mu}) = \prod_{k=1}^K \mu_k^{x_k} $$

ここでN個の独立な観測値$\bm{x_1}, ... \bm{x_N}$を観測したとする.

この時の尤度関数は

$$ \prod_{n=1}^N\prod_{k=1}^K\mu_k^{x_{nk}} = \prod_{k=1}^K\mu_k^{(\sum_{n=1}^Nx_{nk})} = \prod_{k=1}^K\mu_k^{m_k} $$

となる.

ベルヌーイ分布と同様に$m_k$が十分統計量となっている.

ラグランジュ乗数法を用いて対数尤度関数を最大化でき, その結果は

$$ \mu_k^{ML} = \frac{m_k}{N} $$

となる.

また, 1回試行ではなくN回試行の場合もベルヌーイ分布と二項分布の関係のように考えることができ,

$$ \frac{N!}{m_1!m_2!...m_k!}\prod_{k=1}^K \mu_k^{m_k} $$

となる. これは多項分布と呼ばれる.

\subsection{ディリクレ分布}

多項式分布の事前分布を導入する.

多項式分布の形から共益分布は

$$ p(\bm{\mu})|\bm{\alpha}) \propto \prod_{k=1}^K \mu_k^{\alpha_k - 1} $$

これを正規化すると

$$ Dir(\bm{\mu}|\bm{\alpha}) = \frac{\Gamma(\alpha_0)}{\Gamma(\alpha_1)....\Gamma(\alpha_K)}\prod_{k=1}^K \mu_k^{\alpha_k - 1} $$

事後分布を計算したいので事前分布と尤度関数の積を取ると

\begin{eqnarray*}
    p(\bm{\mu}|D, \alpha) &\propto& \frac{N!}{m_1!m_2!...m_k!}\prod_{k=1}^K \mu_k^{m_k} \times \frac{\Gamma(\alpha_0)}{\Gamma(\alpha_1)....\Gamma(\alpha_K)}\prod_{k=1}^K \mu_k^{\alpha_k - 1} \\
    &\propto& \prod_{k=1}^K \mu_k^{m_k} \times \prod_{k=1}^K \mu_k^{\alpha_k - 1} \\
    &\propto& \prod_{k=1}^K \mu_k^{\mu_k + \alpha_k - 1}
\end{eqnarray*}

この事後分布はまた新たにディリクレ分布の事前分布として使用できる形になっていることがわかる.
また, $\alpha$は$m$と一緒に指数部分にあるため, $x_k=1$の有効観測数となっている.

\section{ガウス分布}



\newpage